%=======================================================================
% A listing mode for ssr Gallina (Assia Mahboubi 2007)

\lstdefinelanguage{SSR} {

% Anything between $ becomes LaTeX math mode
mathescape=true,
% Comments may or not include Latex commands
texcl=false,


% Vernacular commands
morekeywords=[1]{
Section, Module, End, Require, Import, Export,
Variable, Variables, Parameter, Parameters, Axiom, Hypothesis, Hypotheses,
Notation, Infix, Local, Tactic, Reserved, Scope, Open, Close, Bind, Delimit,
Definition, Let, Ltac, Fixpoint, CoFixpoint, Morphism, Relation,
Implicit, Arguments, Set, Unset, Contextual, Strict, Prenex,
Implicits, Types,
Inductive, CoInductive, Record, Structure, Canonical, Coercion,
Theorem, Lemma, Corollary, Proposition, Fact, Remark, Example,
Proof, Goal, Save, Qed, Admitted, Defined, Hint, Resolve, Rewrite, View,
Search, Show, Print, Printing, All, Graph, Projections,
inside, outside, Locate, Eval, Compute,
About, Check},


% Gallina
morekeywords=[2]{forall, exists2, fun, fix, cofix, struct,
      match, with, end, as, in, return, let, if, of, is, isn't, then, else,
      for, nosimpl, where},
% 'of' is removed from this list as it occurs in comments.

% Sorts
morekeywords=[3]{Type, Prop},

% Various tactics, some are std Coq subsumed by ssr, for the manual purpose
morekeywords=[4]{
         pose, set, move, case, elim, apply, clear, exists,
            hnf, intro, intros, generalize, rename, pattern, after,
            destruct, induction, using, refine, inversion, injection,
         rewrite, congr, unlock, compute, ring, field,
            replace, fold, unfold, change, cutrewrite, simpl,
         have, suff, wlog, suffices, without, loss, nat_norm,
         gen, generally, abstract,
            assert, cut, trivial, revert, bool_congr, nat_congr, vm_compute,
         symmetry, transitivity, auto, split, left, right,
         autorewrite, lia, intlia},

% Terminators
morekeywords=[5]{
         by, done, exact, reflexivity, tauto, romega, omega,
         assumption, solve, contradiction, discriminate, admit},


% Control
morekeywords=[6]{do, last, first, try, idtac, repeat},

% Various symbols
% For the ssr manual we turn off the prettyprint of formulas
 literate=
%       {->}{{$\rightarrow\ $}}2
%       {->}{{\tt ->}}3
%	{<-}{{$\leftarrow\,$}}2
%       {<-}{{\tt <-}}2
%       {>->}{{$\mapsto$}}3
%       {<}{{$<\;\;$}}1
%       {<=}{{$\leq\;\;$}}1
%       {>=}{{$\geq\;\;$}}1
%       {<>}{{$\neq$}}1
%        {/\\}{{$\wedge\;\;$}}2
%        {\\/}{{$\vee\;\;$}}2
%       {^~}{{\textvisiblespace$\;\;$}}2
%       {~~}{{$\lnot\;\;$}}2
%       {<->}{{$\leftrightarrow\;$}}3
%       {<=>}{{$\Leftrightarrow\;$}}3
%       {:nat}{{$~\in\mathbb{N}$}}3
%	{fforall\ }{{$\forall_f\,$}}1
	{forall\ }{{$\forall\ \!$}}1
%	{exists\ }{{$\exists\ \!$}}1
%       {MAPLE}{{\raisebox{-0.8cm}[0cm][0cm]{\hspace{18mm}\includegraphics[width=1cm]{maple}}}}1
%	{^2}{{$^2$}}1
	{^3}{{$^3$}}1
	{^4}{{$^4$}}1
	{^5}{{$^5$}}1
   {^6}{{$^6$}}1
   {`}{{\textasciigrave}}1
%       {negb}{{$\neg$}}1
%       {spp}{{:*:\,}}1
%       {~}{{$\sim$}}1
%       {\\in}{{$\in\;$}}1
%       {/\\}{$\land\,$}1
%       {:*:}{{$*$}}2
%	{=>}{{$\,\Rightarrow\ $}}1
%       {=>}{{\tt =>}}2
%       {:=}{{{\tt:=}\,\,}}2
%       {==}{{$\equiv$}\,}2
%       {!=}{{$\neq$}\ }2
%       {_cf0}{{$_{{0}}$}}1
%       {_cf1}{{$_{{1}}$}}1
%       {_cf2}{{$_{{2}}$}}1
%       {_cf0_0}{{$_{{00}}$}}1
%       {^-1}{{$^{-1}$}}1
%       {elt'}{elt'}1
%       {=}{{\tt=}\,\,}2
%       {+}{{\tt+}\,\,}2
,

% Comments delimiters, we do turn this off for the manual
%comment=[s]{(*}{*)},

% Spaces are not displayed as a special character
showstringspaces=false,

% String delimiters
morestring=[b]",

% Size of tabulations
tabsize=3,

% Enables ASCII chars 128 to 255
extendedchars=true,

% Case sensitivity
sensitive=true,

% Automatic breaking of long lines
breaklines=true,

% Default style fors listings
basicstyle=\ttfamily\footnotesize,

% Position of captions is bottom
captionpos=b,

% Full flexible columns
columns=[l]fullflexible,

% do not eat spaces after) in inline
keepspaces=true,

% Style for (listings') identifiers
identifierstyle={\ttfamily\color{black}},
% Note : highlighting of Coq identifiers is done through a new
% delimiter definition through an lstset at the beginning of the
% document. Don't know how to do better.

% Style for declaration keywords
keywordstyle=[1]{\ttfamily\color{dkviolet}},

% Style for Gallina keywords
keywordstyle=[2]{\ttfamily\color{dkgreen}},

% Style for sorts keywords
keywordstyle=[3]{\ttfamily\color{lightblue}},

% Style for tactics keywords
keywordstyle=[4]{\ttfamily\color{dkblue}},

% Style for terminators keywords
keywordstyle=[5]{\ttfamily\color{red}},


%Style for iterators
%keywordstyle=[6]{\ttfamily\color{dkpink}},

% Style for strings
stringstyle=\ttfamily,

% Style for comments
%commentstyle=\rmfamily,

}
