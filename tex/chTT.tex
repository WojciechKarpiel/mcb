\chapter{Type Theory}{}
\label{ch:ttch}

%T he authors made the deliberate choice to postpone the presentation of
% the mathematical foundations of the \Coq{} system and the formal
% definition of its language. Instead, the previous chapters have
% dealt with the definition of (typed) programs and on the way these
% programs can be used to describe computable functions and decidable
% predicates. This take on calculations is indeed both at the core of
% the type theory underlying \Coq{} and one of the crucial ingredients
% to the methodology at the base of the \mcbMC{} library. The present chapter is
% devoted to a more in-depth presentation of these mathematical
% foundations, although an exhaustive description shall remain out of
% the scope of the present book.
% For more comprehensive presentations of
% type theory for formalized mathematics, the reader can refer to more
% specialized references like~\cite{ttfp}, or the
% shorter survey in the Handbook of Automated
% Reasoning~\cite[Volume 2, chapter 18]{handbook-ar}.

The \Coq{} proof assistant provides a formal language to describe
mathematical statements and proofs, called
\emph{Gallina} and based on a type theory called the
\emph{Calculus of Inductive
  Constructions} (CIC)~\cite{coquand:huet:88,CoPa89}. In this section, we
provide some hints on the main features of this formalism. The
Calculus of Inductive Constructions is described formally in chapter 4
of the reference manual of \Coq{}~\cite{Coq:manual}. The reference book on
homotopy type theory~\cite{hottbook} describes a very close formalism.



\section{Terms, types, proofs}\label{sec:chi}



\subsection{Propositions as types, proofs as programs}
There exist several flavors of \emph{type theory} that can be used
as a foundational language for mathematics, alternative to the
flavors of set theory usually cited as the formal language of
reference~\cite{bourbaki-sets}.
The main difference between a type-theoretic framework and
a set-theoretic one is the status of the deductive system used to
represent reasoning in a formal way. Informally speaking, a
set-theoretic framework has two stages. First order logic provides
the former layer: it describes the language of logical sentences and how
these statements can be combined and proved. This language is then
used in the second layer, to formulate the axioms of the particular
theory of interest.
On the contrary a type-theoretic framework is monolithic and the
same language of types is used in a uniform way to describe
mathematical objects, statements and proofs. \emph{Logical statements}
are identified with some \emph{types} and their \emph{proofs} with
\emph{terms}, or programs, having this type. Thus proving a statement
consists in fact in constructing a term of the corresponding type. In
set theory, the rules which govern the construction of
grammatically correct statements are rather loose, and the goal of the
game is to construct a proof for a given proposition, using
first-order logic to combine the axioms of the theory.
The analogue of the meta-statement that a given
proposition has a proof is the meta-statement that a given term has a
given type. Such a meta-statement is called a \emph{typing
  judgment}. It is written:
$$\Gamma \vdash t : T$$
which means:
\emph{in the context $\Gamma$, the term $t$ has the type $T$}.
Typing judgments are justified from typing rules, and these
justifications form trees, akin to the proof trees of natural deduction.
Before getting more precise about what contexts, terms and types are,
let us emphasize a couple of important differences with set theory
here: first, the typing judgment features explicit expressions for
the term and the type, whereas the proof of a statement in first-order
logic is seldom mentioned. Second, the term $t$ shall represent an
arbitrary object of the mathematical discourse, like a number or a
group, and not only a proof.


\subsection{First types, terms and computations}\label{ssec:terms}

Every expression we have manipulated so far is a \emph{term} of the
Calculus of Inductive Constructions, and any term of this formalism
has a type in the context in which it occurs.

We assume a countable infinite set of symbols available for names, so
that it is always possible to exhibit a name distinct from a given
arbitrary finite collection of names. This names are used to denote
\emph{variables}, which are among the simplest, atomic terms. The
other atomic terms are the \emph{sorts} \C{Prop} and \C{Type}$_i$, for
$i \in \mathbb{N}$\footnote{Coq has another sort \C{Set} but we do not
  use it here.}.



We % put aside inductive types until section~\ref{ssec:indtypes}, and
define
inductively what the terms of the formalism are. We suppose that we
have a countable infinite set of symbols available for names, so that it is
always possible to exhibit a name distinct from a given arbitrary
finite collection of names. The simplest, atomic, terms are the
\emph{variables}, and the \emph{sorts} \C{Prop} and \C{Type}$_i$, for
$i \in \mathbb{N}$\footnote{Coq has
another sort \C{Set} but we do not use it here.}. We shall also
use constants (for instance to introduce definitions). Constants and
variables are represented using the pre-existing set of names. If \C{x} is a
variable, and \C{T} and \C{U} are two terms, then \C{forall x : T, U} is
also a term. In fact this term will be used as a type, and it
represents a family of types \C{U} indexed by elements of type \C{T}.
Importantly, if \C{x} does not occur in the term \C{U} as a
free variable, then this term is usually denoted \C{T -> U}: it is
used for the type of functions having their argument of type \C{T} and
their results of type \C{U}. For instance, there is no difference
between \C{(nat -> bool)} and \C{(forall x : nat, bool)}, but the
notation. In the case where \C{x} does occur in term \C{U}, the term
\C{forall x : T, U} is called a \emph{dependent type}.
\index[concept]{dependent type}
If \C{x} is a variable and \C{T}, \C{u}
are terms, then \C{(fun x:T => u)} is a term, which represents a function
with argument \C{x} and body \C{u}. If \C{t} and \C{u} are terms then
\C{(t u)} is also a term, which represents the application of (function)
\C{t} to (its argument) \C{u}. Finally \C{let x := t in u} is a term
if \C{t, u} are terms and \C{x} is a variable.

A typing judgment $\Gamma \vdash t : T$ relates a
context $\Gamma$ and two terms $t$ and $T$. As its name suggests,
the context logs all the facts assumed at the moment when the term
occurs. It can be empty but otherwise contains typed variables and
typed definitions. A typed variable is a pair of a name and a
type: for instance, after executing the command \C{Variable n : nat.},
the context contains the pair (\C{n} : \C{nat}). A typed definition is
a triple of a name, a term, and a type: for instance, after executing
the command \C{Definition a : T := t.}, the context contains the
triple (\C{a : T := t}). We have not yet defined what a
type is, because it is a special case of term, defined using typing
judgments. Actually, well-formed contexts and typing judgments are
defined inductively and simultaneously.

Typing judgments with an empty context express the
axiom rules of the formalism, like for instance the types of sorts:
$$\vdash \C{Prop} : \C{Type}_1 \quad \vdash \C{Type}_i : \C{Type}_{i+1}.$$
From now on and as the default display mode of \Coq{} does, we will leave
the index of the sorts \C{Type}$_i$ implicit and just write \C{Type},
as these indices do not play a significant role in what is presented
in this book. A \emph{type} is a term which can be typed by a sort in
a given context:
$$\Gamma \vdash \C{T} : \C{Prop} \quad \textrm{ or}
\quad \Gamma \vdash \C{T} : \C{Type}.$$
Sorts are thus particular types and they are types of types.
A \emph{well-formed context} is either an empty context, or the
extension of an existing context with a pair \C{(x : T)}, where \C{x}
is a fresh name and \C{T} is a type. As expected, a well-formed
context provides a type to the variables it contains:
$$\textrm{if } \Gamma \textrm{ is well-formed  and }
\C{(x : T)} \in \Gamma \textrm{ then }
\Gamma \vdash \C{x} : \C{T}.$$
Contexts not only store global assumptions: they are also used to
forge more complex typing judgments, for more complex types. For
instance, the following rule explains how to construct non-atomic
types:
$$\textrm{if }\quad \Gamma \vdash T : \C{Type} \quad\textrm{ and }\quad
\Gamma, x : T \vdash U : \C{Type} \quad\textrm{ then }\quad
\Gamma \vdash \forall x : T, U : \C{Type}.$$
The term $ \forall x : T, U : \C{Type}$ is the type of functions, as
described by the rule:
$$\textrm{if } \ \Gamma \vdash \forall x : T, U : \C{Type}
\ \textrm{ and }\
\Gamma, x : T \vdash t : U
\ \textrm{ then }\
\Gamma \vdash \C{fun x => t} : \forall x : T, U
$$
and the application of a function to an argument is well-typed only if
their types agree:
$$\textrm{if } \ \Gamma \vdash u : \forall x : T, U
\ \textrm{ and }\
\Gamma \vdash t : T
\ \textrm{ then }\
\Gamma \vdash u\ t : U[t/x]
$$
where $U[t/x]$ denotes the term obtained by substituting
the variable $x$ by the term $t$ in the term $U$.

Consider the simple tautology $A \rightarrow A$.
The implication symbol~$\to$ can be understood as the logical symbol
of implication but also as the symbol we used
to represent the type of functions, in the non-dependent case. The
term \C{(fun x : A => x)} represents the identity function for terms
of type $A$ and is a program of type $A \rightarrow A$. If we consider
that $A$ is a proposition, and that terms of type $A$ are proofs of
$A$, this function is the identity function on proofs of $A$.
More generally, the logical rules of implicative and universal
statements can be decorated with terms, so that proofs of implications
can be seen as functions transforming an arbitrary proof of the
premise into a proof of the conclusion. The introduction rules
respectively for the implication and the universal quantification can
be written:

\begin{center}
\AxiomC{$A$}
\noLine
\UnaryInfC{$\vdots$}
\noLine
\UnaryInfC{$B$}
\RightLabel{$\to_I$}
\UnaryInfC{$A \to B$}
\DisplayProof
\hspace{1cm}
\AxiomC{$B$}
\RightLabel{$\forall_I$ ($x$ fresh)}
\UnaryInfC{$\forall x, B$}
\DisplayProof
\end{center}
% [DG] What do you mean by "fresh" here? $x$ can definitely appear
% in $B$. And shouldn't $x$ be typed in the context?
This presentation, in the style of natural deduction, means that in
order to prove $A \to B$ one can prove $B$ under the
assumption $A$, and in order to prove $\forall x,B$ one can prove $B$
for a fresh variable $x$. Let us now annotate these rules with terms
whose typing rules coincide with these proof rules:

\begin{center}
\AxiomC{\C{x : }$~A$}
\noLine
\UnaryInfC{$\vdots$}
\noLine
\UnaryInfC{\C{b : }$~B$}
\RightLabel{$\to_I$}
\UnaryInfC{\C{(fun x : A => b) : }$~A \to B$}
\DisplayProof
\hspace{1cm}
\AxiomC{\C{b : }$~B$}
\RightLabel{$\forall_I$}
\UnaryInfC{\C{(fun x : A => b) : }$~\forall x, B$}
\DisplayProof
\end{center}

Actually, the term \C{(fun .. => ..)} serves as a proof
for both rules and in both cases the term \C{b} is a proof of $B$.
The only difference is that $x$ is allowed to occur in $B$ only in the
second rule.


We now play the same game with the corresponding elimination rules:


\begin{center}
\AxiomC{$A \to B$}
\AxiomC{$A$}
\RightLabel{$\to_E$}
\BinaryInfC{$B$}
\DisplayProof
\hspace{1cm}
\AxiomC{$\forall x, B$}
\RightLabel{$\forall_E$ ($t$ is a term)}
\UnaryInfC{$B[t/x]$}
\DisplayProof
\end{center}

This time function application serves as a proof in both cases:

\begin{center}
\AxiomC{\C{f : }$~A \to B$}
\AxiomC{\C{t : }$~A$}
\RightLabel{$\to_E$}
\BinaryInfC{\C{(f t) :}$~B$}
\DisplayProof
\hspace{1cm}
\AxiomC{\C{f : }$~\forall x : A, B$}
\AxiomC{\C{t : }$~A$}
\RightLabel{$\forall_E$}
\BinaryInfC{\C{(f t) : }$~B[t/x]$}
\DisplayProof
\end{center}

Universal quantification is not only useful to form types that
play the role of propositions.  It can come handy to form data types
too.  In the second part of this book (section~\ref{sec:matrix})
we shall see how the matrix data type
and the type of its multiplication function benefit from it.

\begin{coq}{}{}
matrix : Type -> nat -> nat -> Type.
mulmx : forall R : Type, forall m n p : nat,
  matrix R m n -> matrix R n p -> matrix R m p.
\end{coq}

This time the data type of matrices exposes the size, and matrix multiplication
can be given a type that rules out incompatible matrices and also describes

the size of the resulting matrix in terms of the size of the input
ones.\footnote{To be precise, \C{mulmx} also needs to take in input operations to
add and multiply elements of \C{R}.  Such detail plays no role in the current
discussion.}

This is in fact a classic example of what is called a \emph{dependent type}: a
data type depending on data, two natural numbers here.
\index[concept]{dependent type}

Note that in this formalism, quantification can range on the proofs of
a given statement, just like it can range on numbers. In this way, a
statement may express
a property shared by the proofs of a given statement. Such capability
finds a rare, but
crucial use in \mcbMC{} that we discuss in chapter~\ref{ch:sigmabool}:
All the proofs of a given equality statement between two booleans are
indistinguishable.

\begin{coq}{name=hedberg}{}
Lemma bool_irrelevance (P Q : bool) : forall e1 e2 : P = Q, e1 = e2.
\end{coq}


The proofs-as-programs correspondence has a visible impact in the proofs
part of the \mcbMC{} library.  In particular, quantified lemmas, being programs,
can be instantiated by simply passing arguments to them.  Exactly as one can
pass \C{3} to \C{addn} and obtain \C{(addn 3)}, the function adding three, one
can ``pass'' \C{3} to the lemma \C{addnC} and obtain a proof of the statement
\C{(forall y, 3 + y = y + 3)}.  Remark that the argument passed to \C{addnC}
shows up in the type of the resulting term \C{(addnC 3)}:  The type of the
\C{addnC} program depends on the value the program is applied to.  That is the
difference between the \emph{dependent function space} ($\forall$)
and the standard function space ($\to$).
\index[concept]{dependent function space}

\mantra{Lemma names can be used as functions, and you can pass
arguments to them.
For example,
\C{(addnC 3)} is a proof that \C{(forall y, 3 + y = y + 3)}, and
\C{(prime_gt0 p_pr)} is a proof that \C{(0 < p)} whenever
\C{(p_pr : prime p)}.}

We refer the reader
to the reference manual of \Coq{}~\cite{Coq:manual} for the other
rules of the system, which are variants of the ones we presented or
which express subtleties of the type system that are out of the scope of
the present book, like the difference between the sorts \C{Prop} and
\C{Type}.


We conclude this section by explaining a last rule of CIC, called the
\emph{conversion rule}, which describes the status of computation in this
framework. Computation is modeled by rewrite rules explaining how to
apply functions to their argument. For instance, the so-called
$\beta$-reduction rule rewrites an application of the form
$(\C{fun x => t}) u$ into $t[u/x]$: the formal argument
$x$ is substituted by the actual argument $u$ in the body $t$ of the
function. A similar computation rule models the computation of a term of
the shape \C{(let x := u in t)} into $t[u/x]$. Two terms
$t_1$ and $t_2$ that are equal modulo computation rules are said to be
\emph{convertible}, and these terms are indistinguishable to the type
\index[concept]{convertibility}
system: If $T_1$ and $T_2$ are two convertible types and if a term $t$
has type $T_1$ in the context $\Gamma$, then $t$ also has type $T_2$
in the context $\Gamma$. This is the feature of the formalism that we
have used in section~\ref{ssec:proofcomp}: The proofs of the
statements \C{2 + 1 = 3} and \C{3 = 3} are the same, because the terms
\C{2 + 1 = 3} and \C{3 = 3} are convertible.  In chapter~\ref{ch:prog} and~\ref{ch:proofs}
we used boolean programs to express predicates and connectives exactly
to take advantage of convertibility: Also the compound
statement
\C{(2 != 7 && prime 7)} is convertible to \C{true}.
Finally, as illustrated in
section~\ref{sec:symcomp}, computation is not limited to terms without
variables: The term \C{(isT : true)}\footnote{In other words \C{isT} is a proof of all statements that are trivial by computation. We invite the reader that finds the writing \C{(isT : true)} ill typed to peek ahead to section~\ref{sec:coercions}.} is a valid proof of
\C{(0 < n.+1)}, as well as a proof of \C{(0 != p.+1)}.
\index[coq]{\C{isT}}

One can also benefit from convertibility whenever one applies a lemma;
otherwise said, whenever the $\to_E$ and $\forall_E$ rules apply.
These rules check that the argument has
the ``right type'', i.e. the type of
the premise or of the  bound variable respectively.
At the cost of being more verbose we could have made explicit
in, say, the typing for $\to_E$
that
types are compared modulo computation as follows:
\begin{center}
\AxiomC{\C{f : }$~A \to B$}
\AxiomC{\C{a : }$~A'$}
\AxiomC{$A \equiv A'$}
\RightLabel{$\to_E$}
\TrinaryInfC{\C{(f a) :}$~B$}
\DisplayProof
\end{center}
where $A \equiv A'$ denotes that $A$ and $A'$ are convertible.

\subsection{Inductive types}\label{ssec:indtypes}

The formalism described in section~\ref{ssec:terms} is extended by
the possibility of introducing
\emph{inductive definitions}~\cite{CoPa89, Moh93}. We only provide a
very brief overview of this subtle feature and again refer the reader
to the reference manual for a precise definition.
In chapter~\ref{ch:prog} we have used several examples of inductive
data types like the data type \C{nat} and its constructors \C{O} and
\C{S}, or the data type \C{option}, or various others.
An inductive definition simultaneously introduces
several new objects into the context: a new type, in a given sort, and
new terms for the constructors, with their types. For instance, the
command:

\begin{coq}{}{}
Inductive nat : Type := O : nat | S (n : nat).
\end{coq}
introduces a new type \C{nat : Type} and two new terms \C{(O : nat)} and
\C{(S : nat -> nat)}. Constructors are functions, possibly of zero
arguments like \C{O}, and the codomain of a constructor of the
inductive type \C{T} is always \C{T}. An inductive type may occur as
the type of certain arguments of its constructors, like in the case of
\C{S}. The only way to construct an inhabitant of an
inductive type is to apply a constructor to sufficiently many
arguments. Constructors are by definition injective functions.
Moreover, two distinct constructors construct distinct
terms; this is why a function with an argument of an inductive type can be
described by pattern matching on this argument. For instance, in
chapter~\ref{ch:prog}, we have defined the predecessor function by:

\begin{coq}{name=predn}{}
Definition predn n := if n is p.+1 then p else n.
\end{coq}
where \C{if ... then ... else ...} is a notation for the special case of
pattern matching with only two branches and one pattern. The definition of
the terms of the formalism is in fact extended with the
\C{match  ... with ... end} construction described in
section~\ref{ssec:nat}. A special reduction rule expresses that
pattern matching a term which features a certain constructor in head
position reduces to the term in the corresponding branch of the case
analysis.

The definition of the terms of CIC also includes so-called
\emph{guarded fixpoints}, which represent functions with a recursive
definition. We have used these fixpoints in chapter~\ref{ch:prog}, for
instance when defining the addition of two natural numbers as:

\begin{coq}{name=add_redef}{}
Fixpoint addn n m :=
  match n with
  | 0 => m
  | p.+1 => (addn p m).+1
  end.
\end{coq}
Note that guarded fixpoints always terminate, as a non-terminating
term would allow proofs of absurdity (see section~\ref{ssec:indreason}).

\subsection{More connectives}\label{ssec:moreconns}
With inductive definitions it is possible to describe more data
structures than the mere functions %of the formalism
described in
section~\ref{ssec:terms}. These data structures and their typing rules
are used to model logical connectives, like functions were used to
model the proofs of implications and universal quantifications.

For instance, the introduction and elimination rule of the conjunction
connective are:

\begin{center}
\AxiomC{$A$} \AxiomC{$B$}
\RightLabel{$\wedge_I$}
\BinaryInfC{$A \wedge B$}
\DisplayProof
\hspace{1cm}
\AxiomC{$A \wedge B$}
\RightLabel{$\wedge_E$ (left)}
\UnaryInfC{$A$}
\DisplayProof
\end{center}
The first rule reads: to prove $A \wedge B$ one needs to prove both
$A$ and $B$.  The second states that one proves $A$ whenever one is
able to prove the stronger statement $A \wedge B$.

In \Coq{} this connective is modeled by the following
inductive definition:

\begin{coq}{name=And}{}
Inductive and (A B : Prop) : Prop := conj (pa : A) (pb : B).
Notation "A /\ B" := (and A B).
\end{coq}
\index[vernac]{\C{Inductive}}

Remark that the ``data'' type \C{and} is tagged as \C{Prop}, i.e.,  we declare
the intention to use it as a logical connective rather than a data type.  The
single constructor \C{conj} takes two
arguments: a proof of \C{A} and a proof of \C{B}.
Moreover \C{and} is polymorphic:
\C{A} and \C{B} are parameters standing for arbitrary propositions.
As a consequence it can model faithfully the logical rule \C{$\wedge_I$}.

Note that the definition of the pair data type,
in section~\ref{sec:othercontainers}, is almost identical to the one
of \C{and}:

\begin{coq}{}{}
Inductive prod (A B : Type) := pair (a : A) (b : B).
\end{coq}
The striking similarity illustrates how programs (and data) are used in \Coq{}
to represent proofs.

Pattern matching provides a way to express the elimination rule for
conjunction as follows:

\begin{coq}{name=Ande1}{}
Definition proj1 A B (p : A /\ B) : A :=
  match p with conj a _ => a end.
\end{coq}
\index[coq]{\C{conj}}

Now recall the similarity between $\to$ and $\forall$, where the former is the
simple, non-dependent case of the latter.  If we ask for the type of
the \C{conj} constructor:

\begin{coq-left}{name=Ande1}{width=4cm}
About conj.
\end{coq-left}
\begin{coqout-right}{}{width=8cm}
conj: forall A B : Prop, A -> B -> A /\ B
\end{coqout-right}

we may wonder what happens if the type of the second argument (i.e., \C{B})
becomes dependent on the value of the first argument (of type \C{A}).
What we obtain is actually the inductive definition corresponding to the
existential quantification.

\begin{coq}{name=Ande1}{}
Inductive ex (A : Type) (P : A -> Prop) : Prop :=
  ex_intro (x : A) (p : P x).
Notation "'exists' x : A , p" := (ex A (fun x : A => p)).
\end{coq}
\index[coq]{\C{ex_intro}}

As  \C{ex\_intro} is the only constructor of the \C{ex} inductive
type, it is the only means to prove a statement like
\C{(exists n, prime n)}.  In such a --- constructive --- proof, the first
argument would be a number
\C{n} of type \C{nat} while the second argument would be a proof \C{p} of type
\C{(prime n)}.  The parameter \C{P} causes the dependency of the
second component of the pair on the first component. It is a function
representing an arbitrary predicate over a term of
type \C{A}.  Hence \C{(P x)} is the instance of the predicate, for \C{x}.  E.g.,
the predicate of being an odd prime number is expressed as
\C{(fun x : nat => (odd x) && (prime x))}, and the statement expressing the
existence of such a number is

\C{(ex nat (fun x : nat => (odd x) && (prime x)))}, which (thanks
to the \C{Notation} mechanism of \Coq{}) is parsed and

printed as the more familiar \C{(exists x : nat, odd x && prime x)}.

It is worth summing up the many features of type theory that
intervene in the type of \C{ex} and \C{ex\_intro}:

\begin{coq}{}{}
ex : forall A : Type, (A -> Prop) -> Prop.
ex_intro : forall A : Type, forall P : A -> Prop, forall a : A, P a -> ex A P.
\end{coq}

Both \C{ex} and \C{ex_intro} are parameterized by \C{(A : Type)}: as
we have seen in chapter~\ref{ch:prog}, the \C{(forall A : Type)}
quantification indicates that \C{ex} are \C{ex_intro} \C{polymorphic}
constants, that can be instantiated for any type. Both \C{ex} and
\C{ex_intro} also have a parameter of type  \C{(A -> Prop)}: since it
does not appear in the rest of the type of \C{ex}, this parameter is
not named and the type uses the arrow syntax instead of a more verbose
\C{forall A : Type, forall P : A -> Prop, Prop}. This parameter is
indicated by a so-called \emph{higher-order} quantification, because
the parameter has an arrow type. Constants  \C{ex} and \C{ex_intro}
can thus be  be specialized to any predicate \C{P}, so that the \C{ex}
inductive declaration can be used on any formula. Finally, the
\C{ex_intro} constant features a last, inner-most \C{forall a : A}
quantifier, which binds a term variable \C{a} representing the witness
of the existential statement for \C{P}.\\

Let us also observe the inductive definition of the disjunction \C{or}
and its two constructors \C{or\_introl} and \C{or\_intror}.

\begin{coq}{name=Or}{}
Inductive or (A B : Prop) : Prop := or_introl (a : A) | or_intror (b : B).
Notation "A \/ B" := (or A B).
\end{coq}

The elimination rule can again be expressed by pattern matching:

\begin{coq}{name=Or}{}
Definition or_ind (A B P : Prop)
  (aob : A \/ B) (pa : A -> P) (pb : B -> P) : P :=
  match aob with or_introl a => pa a | or_intror b => pb b end.
\end{coq}
\index[coq]{\C{or_introl}}
\index[coq]{\C{or_intror}}

The detail worth noting here is that the pattern match construct has two
branches, and each branch represents a distinct sub proof.  In this
case, in order to prove \C{P} starting from \C{A \\/ B},
one has to deal with all cases: i.e., to prove \C{P} under the
assumption \C{A}, and to prove \C{P}
under the assumption \C{B}.

Usual constants and connectives such as $\top$, $\bot$ and $\neg$
can be defined as follows.

\begin{coq}{name=TrueFalse}{}
Inductive True : Prop := I.
Inductive False : Prop := .
Definition not (A : Prop) := A -> False.
Notation "~ A" := (not A).
\end{coq}
\index[coq]{\C{True}}
\index[coq]{\C{False}}
\index[coq]{\C{I}}
\index[coq]{\C{not}}

Hence, in order to prove \C{True}, one just has to apply the
constructor \C{I}, which requires no arguments.
So proving \C{True} is trivial, and as a consequence eliminating it
provides little help (i.e., no extra knowledge is obtained by pattern matching
over \C{I}).  Contrarily, it is impossible to prove \C{False}, since it has no
constructor, and pattern matching on \C{False} can inhabit any type, since no
branch has to be provided:

\begin{coq}{name=exfalso}{}
Definition exfalso (P : Prop) (f : False) : P :=
  match f with end.  (* no constructors, no branches *)
\end{coq}

The only base predicate we still haven't described is equality.  The reason we
left it as the last one is that it has a tricky nature.  In particular,
equality, as we have seen in the previous chapters, is an open notion
in the following sense.  Terms that compute to the same syntactic expression
are considered as equal, and this is true for any program the user may write.
Hence such notion of equality needs to be somewhat primitive, as
\C{match} and \C{fun} are.  One also expects such notion to come
with a substitutivity property: replacing equals by equals must be licit.

The way this internal notion is exposed is via the concept of index
on which an inductive type may vary.

\begin{coq}{name=Eq}{}
Inductive eq (A:Type) (x:A) : A -> Prop := erefl : eq A x x.
Notation "x = y" := (@eq _ x y).
\end{coq}
\index[concept]{equality}
\index[coq]{\C{erefl}}

This is the first time we see a function type after the \C{:} symbol
in an inductive type declaration.
The \C{eq} type constructor takes three arguments: a type \C{A} and
two terms of that type (the former is named \C{x}).
Hence one can write \C{(a = b)} whenever \C{a} and \C{b}
have the same type.
The \C{erefl} constructor takes no arguments, as \C{I}, but its type
annotation says it can be used to inhabit only the type \C{(x = x)}.
Hence one is able to prove \C{(a = b)} only when \C{a} and \C{b} are
convertible
(i.e., indistinguishable from a logical standpoint).
Conversely, by eliminating a term
of type \C{(a = b)} one discovers that  \C{a} and \C{b} are
equal and \C{b} can be freely replaced by \C{a}.

\begin{coq}{name=EqInd}{}
Definition eq_ind A (P : A -> Prop) x (px : P x) y (e : x = y) : P y :=
  match e with erefl => px end.
\end{coq}

The notion of equality is one of the most intricate aspects of type
theory; an in-depth study of it is out of the scope of this book.  The interested reader
finds an extensive study of this subject in~\cite{hottbook}.  Later in this
chapter we define and use other inductive types to take advantage
of the ``automatic'' substitution of the implicit equations we see here:
While \C{px} has type \C{(P x)}, it is accepted as an
inhabitant of \C{(P y)} because inside the \C{match} the term \C{y}
is automatically replaced by \C{x}.


%%%%%%%%%%%%%%%%%%%%%%%%%%%%%%%%%%%%%%%%%%%%%%%%%%%%%%%%%%%%
\subsection{Inductive reasoning}\label{ssec:indreason}

In chapter~\ref{ch:prog} we have seen how to build and use (call or destruct)
anonymous functions and data types.  All these
constructions have found counterparts in the Curry-Howard correspondence.
The only missing piece is recursive programs.  For example,
\C{addn} was written by recursion on its first argument, and is a
function taking as input two numbers and producing a third one.
We can write programs by recursion that take as input, among regular  data,
proofs and produce  other proofs as output.  Let's look at the
induction principle for natural numbers through the looking glasses of the
Curry-Howard isomorphism.

\begin{coq}{}{width=3.5cm}
About nat_ind.
\end{coq}
\begin{coqout}{}{width=9cm}
nat_ind : forall P : nat -> Prop,
  P 0 -> (forall n : nat, P n -> P n.+1) -> forall n : nat, P n
\end{coqout}
\C{nat\_ind} is a program that produces a proof of \C{(P n)} for any \C{n},
proviso a proof for the base case \C{(P 0)}, and a proof
of the inductive step \C{(forall n : nat, P n -> P n.+1)}.
Let us write such a program by hand.

\begin{coq}{}{}
Fixpoint nat_ind (P : nat -> Prop)
  (p0 : P 0) (pS : forall n : nat, P n -> P n.+1) n : P n :=
  if n is m.+1 then
    let pm (* : P m *) := nat_ind P p0 pS m in
    pS m pm (* : P m.+1 *)
  else p0.
\end{coq}
\index[coq]{\C{nat\_ind}}

The \Coq{} system generates this program automatically, as soon as
 the \C{nat} data type
is defined.  Recall that recursive functions are checked for termination:
Through the lenses of the proofs-as-programs correspondence, this means
that the induction principle just coded is sound, i.e., based on a
well-founded order relation.
\index[concept]{termination}
\index[concept]{consistency}

If non-terminating functions are not ruled out, it is easy to inhabit
the \C{False} type, even if it lacks a proper constructor.

\begin{coq}{}{}
Fixpoint oops (n : nat) : False := oops n.
Check oops 3.  (* : False *)
\end{coq}
Of course \Coq{} rejects the definition of \C{oops}.  To avoid
losing consistency, \Coq{} also enforces some restrictions on
inductive data types.  For example the declaration of \C{hidden}
is rejected.

\begin{coq}{}{}
Inductive hidden := Hide (f : hidden -> False).
Definition oops (hf : hidden) : False := let: Hide f := hf in f hf.
Check oops (Hide oops).  (* : False *)
\end{coq}
Note how \C{oops} calls itself, as in the previous example,
even if it is not a recursive function.
Such restriction, called
\emph{positivity condition}, is out of scope for this book.
(Roughly speaking, it says that constructors for an inductive data
type can only depend on maps \emph{to} the data type but not on maps
\emph{from} it.)
The interested reader shall refer to~\cite{Coq:manual}.
\index[concept]{positivity}

%%%%%%%%%%%%%%%%%%%%%%%%%%%%%%%%%%%%%%%%%%%%%%%%%%%%%
\section{On the status of Axioms}
\label{sec:EM}

Not all valid reasoning principles can be represented by programs.
For example, excluded middle --- i.e., the claim that
\C{A \\/ ~~ A} for a given logical statement \C{A} ---
can be proved by a program only when
\C{A} is decidable, i.e., when we can write in \Coq{} a program
to \C{bool} that tests if \C{A} holds or not.
Excluded middle, in its generality, can only be \emph{axiomatized},
i.e., assumed globally.

The \mcbMC{} library is axiom free.  This makes the library compatible
with any combination of axioms that is known to be consistent with the
Calculus of Inductive Constructions. Some formal developments in
\Coq{} tend to assume the excluded middle axiom, even if the proofs
formalized do not require it, to avoid (or postpone) proving the
decidability of the predicates at play while still being able to
reason by case analysis on their validity. By contrast, the \mcbMC{}
library provides the decidability proofs when they exist. Moreover, the
boolean reflection methodology, object of chapter~\ref{ch:boolrefl},
provides tools to manipulate decidable predicates in a convenient way,
with the same ease as in a formalism based on a classical logic.

Yet sometimes it is not possible to provide a constructive proof of a
statement. In this case, axioms like the excluded middle axiom can
still be confined into ``boxes''. The purpose of these boxes is to
delimit a local context in which the required axioms are
available. Such a box is typically called a \emph{monad}
in the theory of programming languages. Here is an example, for the
axiom of excluded middle:

\begin{coq}{}{}
Definition classically P : Prop := forall b : bool, (P -> b) -> b.
Lemma classic_EM P : classically (decidable P).
Lemma classicW P : P -> classically P.
Lemma classic_bind P Q :
  (P -> classically Q) -> classically P -> classically Q.
\end{coq}
\index[coq]{\C{classically}}
\index[coq]{\C{classically_EM}}
\index[coq]{\C{classicW}}
\index[coq]{\C{classically_bind}}
% [DG] If we mention monads, let's include their unit.
%assia : TODO
The \C{classically} box can only be opened when the statement to be
proved in the current goal is a boolean, hence an instance of a
decidable predicate. Inside such a box the excluded middle is
made available by combining \C{classic_EM} and \C{classic_bind}.
Nevertheless, when proving a statement that is not a boolean, like
\C{exists n, ...}, one cannot access assumptions in the \C{classically} box.

In other cases, axioms can be avoided by rephrasing the mathematics
in a weaker setting.  A notable example in the \mcbMC{} library
is the construction of the real closure of an Archimedean
field~\cite{DBLP:conf/itp/Cohen12}.
%\marginnote{TODO Cyril: say more about RCF}
